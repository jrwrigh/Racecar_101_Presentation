% Inbuilt themes in beamer
\documentclass[aspectratio=169]{beamer}

% Theme choice:
\usetheme{CambridgeUS}
\usecolortheme{orchid}
\setbeamertemplate{navigation symbols}{} % Remove navigation buttons

%% Get better (serif) math font
\usefonttheme{professionalfonts}

% Title page details:
\title{Racecar 101}
\author{James Wright}
% \date{\today}
% \logo{\large \LaTeX{}}

\begin{document}

% Title page frame
\begin{frame}
    \titlepage
\end{frame}

% Remove logo from the next slides
\logo{}

% Outline frame
\begin{frame}{Outline}
    \tableofcontents
\end{frame}

\section{What makes a car fast?}

\begin{frame}
    \begin{alertblock}{Note}
        This first part is a very simplified breakdown
        \begin{itemize}
            \item It's not the most accurate
            \item It's not to insult anyone's intelligence
        \end{itemize}
        It's simply to not distract from the things that can be easily
        forgotten or muddied.
    \end{alertblock}
\end{frame}

\begin{frame}{What makes a car go fast?}

    \onslide<+-> {
        \begin{equation*}
            \mathrm{Time} = \frac{\mathrm{Distance}}{\mathrm{Velocity}}
        \end{equation*}
    }

    \begin{itemize}
        \onslide<+->{\item To lower time, we need to increase velocity\footnote{Assuming distance is constant}}
        \item<+-> All motorsports have velocity changes during a race
            \begin{itemize}
                \item Excluding top-speed records of course
            \end{itemize}
        \item<+-> Change in velocity is... \onslide<+->{\textbf{Acceleration}}
        \item<+-> To maximize velocity, you must maximize acceleration
            \begin{itemize}
                \item ie. Whatever changes in velocity you make, do them as quickly as possible
            \end{itemize}
        \begin{block}{}<+->
            To make a car \textbf{faster}, you must make the car \textbf{accelerate more}
        \end{block}
    \end{itemize}

\end{frame}

\begin{frame}{What famous equation involves acceleration?}
    \onslide<2->{
        \center Newton's 2nd law!
        \begin{equation*}
            F=ma
        \end{equation*}
    }

    \onslide<3->{
        We care about acceleration, so rearange:
        \begin{equation*}
            a = \frac{F}{m}
        \end{equation*}
    }
\end{frame}

\begin{frame}{How do we maximize acceleration?}
    \begin{equation*}
        a = \frac{F}{m}
    \end{equation*}

    \begin{block}{Decrease Mass}<+->
        \begin{itemize}
            \item Make things lighter
        \end{itemize}
    \end{block}

    \begin{block}{Increase Force}<+->
        \begin{itemize}
            \item<+-> Increase the force the tires can apply to the ground
            \item<+-> Increase power output
            \item<+-> Increase braking torque
        \end{itemize}
    \end{block}

    \begin{alertblock}{}<+->
        The latter two hold \textbf{only if the tires can transfer the torque}
    \end{alertblock}
\end{frame}

\begin{frame}{Balancing \(\uparrow\)Force vs \(\downarrow\)Mass}
    \onslide<+->{
        Sometimes \(\uparrow\) mass + \(\uparrow\) force = \(\uparrow\) acceleration
    }
    \begin{exampleblock}{Bigger Engine}<+->
        Increases the total vehicle mass, but increases power output

        Depending on the ratio, can lead to better acceleration.
    \end{exampleblock}

    \vspace{10pt}
    \onslide<+->{
        Sometimes \(\downarrow\) mass + \(\downarrow\) force = \(\uparrow\) acceleration
    }
    \begin{exampleblock}{Smaller/Narrower Tires}<+->
        Decreases total vehicle mass, but decreases total acceleration potential

        Also reduces unsprung mass (improves vehcile handling and response)
    \end{exampleblock}
\end{frame}

\begin{frame}{Longitudinal Acceleration}
    \onslide<+->{
        Simplest acceleration to model:
        \begin{equation*}
            a = \frac{F}{m}
        \end{equation*}

        Tire traction capacity sets upper limit of the acceleration.
    }

    \onslide<+->{Divided into 2 components:}
    \begin{enumerate}
        \item<+-> Braking (negative)
            \begin{alertblock}{Braking should \textbf{ALWAYS} be limited by tire traction}<+->
                \begin{itemize}
                    \onslide<+->{
                        \item This is as much for safety as it is performance
                        \item Ensure that care is capable of absolute maximum braking acceleration
                    }
                \end{itemize}
            \end{alertblock}
        \item<+-> Power (positive)
        \begin{itemize}
            \item<+-> Almost always limited by the power unit (ICE, electric motor, rubber band windup, etc.)
        \end{itemize}
    \end{enumerate}
\end{frame}

\begin{frame}{Lateral Acceleration}

    \onslide<+->{
        Turning causes \textit{Lateral Acceleration}, which is not a change in speed, but of direction:
        \begin{equation*}
            a_\mathrm{lat} = \frac{V^2}{r}
        \end{equation*}
        where \(V\) is velocity, and \(r\) is the turning radius.
    }

    \vspace{7pt}
    \onslide<+-> {
    Plugging back into momentum balance yields:
        \begin{equation*}
             F = m\frac{V^2}{r}  \ \Rightarrow \ V = \sqrt{\frac{Fr}{m}}
        \end{equation*}
    }

    \vspace{-7pt}
    \onslide<+->{
        Therefore given:
        \begin{itemize}
            \item a force, \(F\) (tire traction)
            \item a mass, \(m\) (the car)
            \item and a radius, \(r\) (the track/racing line)
        \end{itemize}
        there is a \textbf{limit to the maximum velocity}
    }
\end{frame}

\begin{frame}{Lateral Acceleration cont.}
    \onslide<+->{
        How do we maximize the velocity? \(V = \sqrt{\frac{Fr}{m}}\)
    }
    \begin{enumerate}
        \item<+-> Decrease mass \(m\)
        \begin{itemize}
            \onslide<+->{
                \item Add lightness
                \item Has compounding affect due to load transfer (discussed later)
            }
        \end{itemize}
        \item<+-> Increase force \(F\)
        \begin{itemize}
            \item<+-> Increase the maximum force the tires can exert
            \item<+-> How?
                \onslide<+->{
                    \begin{itemize}
                        \item Aero downforce
                        \item Different tires
                        \item Suspension design, etc....
                    \end{itemize}
                }
        \end{itemize}
    \end{enumerate}
\end{frame}

\begin{frame}{Quick Review}

    \begin{block}{Higher Acceleration = Faster Car}

    \end{block}

    \begin{table}[]
    \begin{tabular}{l|ll}
                & Limited by                & How to make better? \\ \hline
    Longitudinal & Force (Braking and Power) & Bigger Engine/Brakes \\ \cline{2-3}
    Acceleration & Mass                      & Reduce it \\ \hline
    Lateral      & Force (Tire Traction)     & Increase Grip \\ \cline{2-3}
    Acceleration & Mass                      & Reduce it
    \end{tabular}
    \end{table}

\end{frame}

% Blocks frame
\section{Vehicle Basics}
\begin{frame}{G-G Curve}
    \begin{block}{What about lateral and longitudinal acceleration at the same time?}<+->
        \onslide<+->{
            Answer: look at a G-G curve for the car
        }
    \end{block}
\end{frame}

\end{document}
